% !TeX root = RJwrapper.tex
\title{Connecting R with D3 for dynamic graphics, to explore multivariate data
with tours}
\author{by Michael Kipp, Dianne Cook, Ursula Laa}

\maketitle

\abstract{%
The tourr package in R (REF) has several algorithms and displays for
showing multivariate data as a sequence of low-dimensional projections.
It can display as a movie but has no capacity for interaction, such as
stop/go, change tour type, drop/add variables. The tourrGui (REF)
package provides these sorts of controls, but the interface is
programmed with the dated RGtk2 (REF) package. This work explores using
custom messages to pass data from R to D3 (REF) for viewing, using the
shiny framework.
}

\subsection{Introduction}\label{introduction}

The tour algorithm (Cook et al. \protect\hyperlink{ref-gt_pp}{1995};
Cook et al. \protect\hyperlink{ref-gt_pp_mc}{2007}) is a way of
systematically generating and displaying projections of high-dimensional
spaces in order for the viewer to examine the multivariate distribution
of data. It can do this either randomly, or by picking projections
judged interesting according to some criterion or index function. The
tourr package (Wickham et al. \protect\hyperlink{ref-tourr}{2011})
provides the computing and display in R (R Core Team
\protect\hyperlink{ref-R}{2012},Ihaka and Gentleman
(\protect\hyperlink{ref-ihaka:1996}{1996})) to make several types of
tours, grand, guided, little and local. The projection dimension can be
chosen between one and the number of variables in the data. The display,
though, has no capacity for interaction. The viewer can watch the tour
like a movie, but not pause it and restart, or change tour type, or
number of variables.

These interactive controls were provided with the tourrGui package
(Huang, Cook, and Wickham \protect\hyperlink{ref-tourrGui}{2012}), with
was programmed with the RGtk2 package (Lawrence and Temple Lang
\protect\hyperlink{ref-RGtk2}{2010}). This is not the toolkit of choice
today, and has been superceded with primarily web-capable tools, like
shiny (Chang et al. \protect\hyperlink{ref-shiny}{2017}). To display
dynamic graphics though, still takes some work. This work explores the
use of D3 (Heer \protect\hyperlink{ref-2011-d3}{2011}) as the display
engine.

\subsection{Section title in sentence
case}\label{section-title-in-sentence-case}

This section may contain a figure such as Figure \ref{figure:rlogo}.

\begin{figure}[htbp]
  \centering
  \caption{The logo of R.}
  \label{figure:rlogo}
\end{figure}

\subsection{Another section}\label{another-section}

There will likely be several sections, perhaps including code snippets,
such as:

\begin{Schunk}
\begin{Soutput}
#>  [1]  1  2  3  4  5  6  7  8  9 10
\end{Soutput}
\end{Schunk}

\subsection{Summary}\label{summary}

This file is only a basic article template. For full details of
\emph{The R Journal} style and information on how to prepare your
article for submission, see the
\href{https://journal.r-project.org/share/author-guide.pdf}{Instructions
for Authors}.

\subsection{References}\label{references}

\hypertarget{refs}{}
\hypertarget{ref-shiny}{}
Chang, Winston, Joe Cheng, JJ Allaire, Yihui Xie, and Jonathan
McPherson. 2017. \emph{Shiny: Web Application Framework for R}.
\url{https://CRAN.R-project.org/package=shiny}.

\hypertarget{ref-gt_pp}{}
Cook, Dianne, Andreas Buja, Javier Cabrera, and Catherine Hurley. 1995.
``Grand Tour and Projection Pursuit.'' \emph{Journal of Computational
and Graphical Statistics} 4 (4): 155--72.

\hypertarget{ref-gt_pp_mc}{}
Cook, Dianne, Andreas Buja, Eun-Kyung Lee, and Hadley Wickham. 2007.
``Grand Tours, Projection Pursuit Guided Tours and Manual Controls.''

\hypertarget{ref-2011-d3}{}
Heer, Michael Bostock AND Vadim Ogievetsky AND Jeffrey. 2011. ``D3:
Data-Driven Documents.'' \emph{IEEE Trans. Visualization \& Comp.
Graphics (Proc. InfoVis)}. \url{http://vis.stanford.edu/papers/d3}.

\hypertarget{ref-tourrGui}{}
Huang, Bei, Dianne Cook, and Hadley Wickham. 2012. ``TourrGui: A
gWidgets Gui for the Tour to Explore High-Dimensional Data Using
Low-Dimensional Projections.'' \emph{Journal of Statistical Software} 49
(6): 1--12.

\hypertarget{ref-ihaka:1996}{}
Ihaka, Ross, and Robert Gentleman. 1996. ``R: A Language for Data
Analysis and Graphics.'' \emph{Journal of Computational and Graphical
Statistics} 5 (3): 299--314.

\hypertarget{ref-RGtk2}{}
Lawrence, Michael, and Duncan Temple Lang. 2010. ``RGtk2: A Graphical
User Interface Toolkit for R.'' \emph{Journal of Statistical Software}
37 (8): 1--52. \url{http://www.jstatsoft.org/v37/i08/}.

\hypertarget{ref-R}{}
R Core Team. 2012. \emph{R: A Language and Environment for Statistical
Computing}. Vienna, Austria: R Foundation for Statistical Computing.
\url{http://www.R-project.org/}.

\hypertarget{ref-tourr}{}
Wickham, Hadley, Dianne Cook, Heike Hofmann, and Andreas Buja. 2011.
``Tourr: An R Package for Exploring Multivariate Data with
Projections.'' \emph{Journal of Statistical Software} 40 (2): 1--18.

\address{%
Michael Kipp\\
Monash University\\
Department of Econometrics and Business Statistics\\
}
\href{mailto:mkipp271@gmail.com}{\nolinkurl{mkipp271@gmail.com}}

\address{%
Dianne Cook\\
Monash University\\
Department of Econometrics and Business Statistics\\
}
\href{mailto:dicook@monash.edu}{\nolinkurl{dicook@monash.edu}}

\address{%
Ursula Laa\\
Monash University\\
Department of Physics\\
}
\href{mailto:ursula.laa@monash.edu}{\nolinkurl{ursula.laa@monash.edu}}

